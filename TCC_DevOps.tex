%
% Exemplo LaTeX de artigo UNISINOS
%
% Elaborado com base nas orientações dadas no documento
% ``GUIA PARA ELABORAÇÃO DE TRABALHOS ACADÊMICOS''
% disponível no site da biblioteca da Unisinos.
% http://www.unisinos.br/biblioteca
%
% Os elementos textuais abaixo são apresentados na ordem em que devem
% aparecer no documento.  Repare que nem todos são obrigatórios - isso
% é devidamente indicado em cada caso.
%
% Comentários abaixo colocados entre aspas (`` '') foram
% extraídos diretamente do documento da biblioteca.
%
% Este documento é de domínio público.
%

%=======================================================================
% Declarações iniciais identificando a classe de documento e
% selecionando alguns pacotes adicionais.
%
% As opções disponíveis (separe-as com vírgulas, sem espaço) são:
% - twoside: Formata o documento para impressão frente-e-verso
%   (o default é somente-frente)
% - english,brazilian,french,german,etc.: idiomas usados no documento.
%   Deve ser colocado por último o idioma principal.
%=======================================================================
\documentclass[twoside,english,brazilian]{UNISINOSartigo}
\usepackage[utf8]{inputenc} % charset do texto (utf8, latin1, etc.)
\usepackage[T1]{fontenc} % encoding da fonte (afeta a sep. de sílabas)
\usepackage{graphicx} % comandos para gráficos e inclusão de figuras
\usepackage{bibentry} % para inserir refs. bib. no meio do texto

%=======================================================================

\unisinosbst
\usepackage{xcolor}
%\usepackage[alf]{abntcite}

%=======================================================================
% Início do documento.
%=======================================================================
\begin{document}
% Diferentemente do normal, os comandos a seguir devem vir aqui mesmo,
% e não antes do \begin{document} onde seria o lugar deles. 
\titulo{Quem são os profissionais DevOps e quais são as suas habilidades?}
\autor{Kim Bemfica Agliardi\footnote{Aluno do curso de Ciência da Computação.  Email: kim.agliardi@gmail.com}}
\autor{Christopher da Rosa Pohlmann \footnote{Orientador, professor da Unisinos, Mestre em Engenharia de Produção e Sistemas pela Universidade do Vale do Rio dos Sinos(2009)  Email: chrisp@unisinos.br}}

%=======================================================================
% Resumo em Português.
%
% A recomendação é para 150 a 250 palavras.
%=======================================================================
\begin{abstract}
Ao longo dos últimos anos a cultura de DevOps tem ganhado força no mercado de TI. O seu principal objetivo é o alinhamento entre as equipes de desenvolvimento e operação, para que juntos os times realizem entregas mais rápidas e de qualidade, por meio de ferramentas e responsabilidades previamente definidas. Trata-se de uma cultura de colaboração entre às equipes. Este estudo tem como objetivo realizar um panorama sobre os profissionais que estão trabalhando em organizações que praticam a cultura DevOps, entendendo qual é o seu nível de experiência no mercado de TI e quais práticas de DevOps estão em uso por estes profissionais, bem como mapear dificuldades encontradas por estes profissionais ao longo de implantações de ferramentas de automação, também pretendemos mapear quais recursos são utilizados na busca de novos conhecimentos por estes profissionais.

\palavraschave{DevOps\@. Infraestrutura como código\@. Metódos Ágeis \@. Entrega Contínua \@. Integração Contínua. \@ }
\end{abstract}

%=======================================================================
% Introdução
%=======================================================================
\section{Introdução}
Ao longo dos últimos anos, foi possível observar um crescimento na importância das áreas de TI, fato que até então, não era verdade.  \citetexto{Beal2009}  lembra que "por muito tempo  a TI foi considerada um "centro de custo" e que em princípio não gerava qualquer retorno para o negócio", dando ênfase os altos valores de investimento em equipamentos e o pouco uso que se fazia destes.  Atualmente, as organizações vêm enfrentando um ambiente extremamente competitivo, inseridas em uma sociedade profundamente afetada pelos novos paradigmas introduzidos pela chamada sociedade da informação. \citep{AudyFreitag08}.
A nova realidade provoca a reorganização intensa da sociedade, gerando modificações nas organizações. Neste contexto, as áreas de sistemas de informação tem se tornado cada vez mais críticas, dado o constante clima de mudança.

Com toda essa competitividade existente no mercado, mudanças de software tornaram-se comuns para as empresas, que até então, não estavam habituadas a realizar mudanças tão constantes em suas aplicações. O aumento no ritmo de entrega de software evidenciou a disparidade existente entre as equipes de desenvolvimento e operações de TI existentes nas organizações. Enquanto equipes de desenvolvimento, de um modo geral, possuem metodologias, frameworks e ferramentas bem estruturadas, que automatizam etapas de desenvolvimento de software e tratam do ciclo de vida de uma aplicação, equipes de operações de TI ainda  não estão tão bem estruturadas, realizando diversas tarefas de maneira manual e com uma baixa padronização. 

Confome \citetexto{Debois2008} observou as equipes de infraestrutura também podem adotar algumas das práticas já existentes e utilizadas por equipes de desenvolvimento, e a partir dos questionamentos levantados pelo artigo "Agile and Operations Infrastructure: How Infra-gile Are You?", 
foi gerada a lista de discussao "Agile-Sysadmin", que pode ser considerada uma semente inicial da cultura DevOps, que prega justamente a adoção de práticas de engenharia de software já consideradas maduras. Também podemos pensar em DevOps como uma evolução natural dos métodos ágeis, que foi estendido e passou à atender também áreas de operação. É importante salientar que diferente do movimento ágil ou ITIL, DevOps não possui uma "metodologia" para implementação, ou até mesmo, um manifesto, o que acaba tornando seu processo de adoção um pouco mais complexo, dado este contexto, diversos estudos buscam uma definição sobre o que é DevOps, quais são seus processos e práticas e quais são as características que as companhias buscam nos profissionais que vão compor estas equipes.

Desta forma, este estudo tem como objetivo analisar quais são as características dos profissionais que estão inseridos em ambientes que promovem a cultura DevOps, entendendo quais são suas habilidades técnicas, grupos de ferramentas utilizadas, experiências e dificuldades encontradas por estes profissionais para sustentar as práticas de automação que são aplicadas neste novo paradigma de trabalho.

A referida análise desdobra-se no seguintes objetivos específicos: (i) Identificar experiências prévias destes profissionais no mercado de TI, (ii) Identificar quais práticas de DevOps estão em uso por parte dos profissionais, (iii) Analisar como os profissionais estão se preparando (estudos) para este novo paradigma, e como buscam informações sobre novas ferramentas / práticas, (iv) Analisar quais são as dificuldades e ganhos encontrados por estes profissionais na adoção deste novo paradigma

O \textbf{segundo capítulo} apresenta o referencial teórico, conceituando em linhas gerais o que é DevOps, quais são as características que esta cultura apresenta e uma breve descrição sobre as práticas utilizadas pelos profissionais.O \textbf{terceiro capítulo} apresenta-se as etapas de pesquisa. No \textbf{quarto capítulo} detalham-se os resultados do survey e por fim, no \textbf{quinto capítulo} seguem as conclusões.
%=======================================================================
% Referencial Teórico
%=======================================================================
\section{REFERENCIAL TEÓRICO}
	Ao longo deste capítulo, apresentam-se os principais conceitos sobre DevOps, como seus princípios, métodos e modelos, bem como a história de como surgiu o movimento.

\subsection{A relação entre Desenvolvimento  e Infraestrutura/Operação}

Com a pressão constante para entrega de valor por meio de software aos clientes, não é incomum que áreas de desenvolvimento e operações entrem em conflito. Enquanto desenvolvedores buscam pôr em produção o mais brevemente possível novas funcionalidades/sistemas, equipes de operação preocupam-se com a estabilidade, o que significa não realizar alterações em sistemas de produção frequentemente.
Segundo \citetexto{HUTTERMANN12}, estes conflitos geram barreiras culturais e organizacionais, que ocasionam as seguintes situações:

\begin{itemize}
\item Como o foco das equipes é diferente, a tensão entre as equipes aumenta e cada uma defende seus interesses individuais, sem pensar no todo;
\item A lacuna destes processo resulta em diferentes abordagens entre Dev/Ops sobre como gerenciar mudanças, colocá-las em produção e mantê-las funcionando;
\item  A origem dessa lacuna começa pelas próprias ferramentas utilizadas pelas equipes, que geralmente são diferentes.
\end{itemize}

Existe um termo na indústria que descreve esse tipo de comportamento como "O muro da confusão", que é causado justamente pela combinação de diferentes motivações (cada equipe defende o seu ponto), processos mal formulados e diferentes ferramentas de trabalho. Como cada equipe acredita estar fazendo o correto pelo negócio, as decisões ocorrem de maneira isolada e este "jogo de responsabilidades" ocorre.

\begin{figure}[h]
    \centering
    \includegraphics[scale=.9]{imagens/wallOfConfusion.jpg}
    \caption{Muro da confusão}
    \label{fig:Muro da confusão}
\end{figure}
É interessante observar que ao longo do tempo o processo de desenvolvimento de software sofreu diversos aprimoramentos, passando por modelos como o cascata (waterfall) até chegar no padrão que é visto hoje, que é baseado em metodologias ágeis. Já áreas de operações geralmente baseiam-se em métodos como ITIL (Information Technology Infrastructure Library), que possui processos processos claros e bem definidos para cada etapa de vida de um serviço. Essa diferença processual segue o mesmo descompasso que é visto no ferramental utilizado pelas equipes, enquanto métodos ágeis visam a entrega de pequenas partes de um sistema de maneira constante, gerando um grande número de alterações em um breve período de tempo por meio de suas "sprints", já ITIL, com o processo de gerênciamento de mudanças, prevê um modelo formal com reuniões e discussões entre às equipes antes da realização de alguma alteração. 

\subsection{Métodos ágeis}
O movimento ágil começou com a escrita do \textbf{Manifesto Ágil} em 2001, e o nome "Agile" foi o nome dado ao conjunto de métodos de desenvolvimento de software que haviam sido desenhados para serem mais leves e flexíveis do que modelos como o cascata.
O que diz o  \textbf{Manifesto Ágil} :  \newline
Estamos descobrindo maneiras melhores de desenvolver 
software, fazendo-o nós mesmos e ajudando outros a 
fazerem o mesmo. Através deste trabalho, passamos a valorizar: 
\begin{itemize}
\item \textbf{Indivíduos e interações} mais que processos e ferramentas;
\item \textbf{Software em funcionamento} mais que documentação abrangente;
\item \textbf{Colaboração com o cliente} mais que negociação de contratos;
\item \textbf{Responder a mudanças} mais que seguir um plano
\end{itemize}
Ou seja, mesmo havendo valor nos itens à direita, valorizamos mais os itens à esquerda. \newline

É importante perceber que o principal objetivo do movimento ágil é enfatizar a colaboração, flexibilidade e como resultado final, diminuir o tempo de entrega de software, diminuir o valor gasto e fornecer um software de qualidade. Porém, os métodos ágeis não definem práticas para a disponibilização do software, o que acaba gerando uma lacuna entre as equipes de desenvolvimento e operação. Apesar do aumento de performance gerado pela adoção de práticas ágeis, existe a demora da fase de implantação do software, que em muitos casos, é causado pelo medo de "quebra" da aplicação à cada nova funcionalidade ou correção integrada a solução, por parte da equipe de operações, que tem como foco a estabilidade da aplicação.
O modelo ágil é próximo ao DevOps tanto no sentido cultural, isto é, o foco está nos indivíduos e na colaboração, quanto na redução de tempo de disponibilização de software. As práticas de integração contínua, entrega contínua, testes contínuos e  infraestrutura como código, seguem justamente a ideia de não aguardar um ciclo inteiro para integrar, testar e disponibilizar, mas sim, integrar as funcionalidades assim que possível.

\subsection{ITIL}
A Information Technology Infrastructure Library (ITIL), é um conjunto de práticas definidas para o gerenciamento de serviços de TI . É uma publicação que contém 5 volumes que descrevem: Processos, procedimentos, tarefas e checklists e é utilizada por organizações para demonstrar conformidade e  medidas para melhoria de serviços. \citetexto{Davis} \newline
Sua primeira versão foi publicada no final dos anos 80, e o número de livros e práticas cresceu com o passar dos anos, sua última publicação (ITIL V3) foi realizada no ano de 2011, e esta versão contém livros sobre: Estratégias de serviço, Design de serviço, Transição de serviço, Operação de serviço e Melhoria continuada. 
Conforme observado por \citetexto{Borangiu2016}, métodos ágeis e ITIL possuem diferenças tanto conceituais como estruturais. Métodos ágeis são vistos como leves e flexíveis, enquanto ITIL é considerado burocrático e processual. Apesar de ambos possuírem o mesmo objetivo que é prover a entrega de valor ao negócio, no entanto, a maneira como isso é realizado é extremamente diferente para cada um destes frameworks. Enquanto métodos ágeis visam entregar rapidamente o que o cliente deseja, ITIL visa entregar serviços de TI estáveis, respeitando um conjunto de níveis de serviço e qualidade previamente acordados. O manifesto ágil favorece as interações informais entre os participantes do projeto, em oposição a um alto formalismo, que é defendido pelo ITIL.
Dado esse contexto, é interessante observar que este é mais um ponto de dissidência presente entre as equipes de desenvolvimento e operações. Enquanto uma visa "fornecer valor" por meio da entrega rápida de funcionalidades (Dev) a outra busca um alto nível de estabilidade e satisfação do cliente (Ops), garantindo boas avaliações de nível de serviço e qualidade.



\subsection{DevOps}
Nesta subseção aborda-se a origem do termo DevOps, suas principais características e práticas.

\subsubsection{A origem da cultura "DevOps"}
As "sementes" da cultura DevOps foram plantadas no ano de \textbf{2008}, pelo artigo publicado pelo belga  Patrick Debois, denominado "Agile and Operations Infrastructure: How Infra-gile Are You?" \citep{Debois2008}. A sua motivação para realização de sua pesquisa foi uma série de frustrações que ocorreram por conta de conflitos entre desenvolvedores e administradores de sistemas, ao longo de uma migração de um datacenter do governo belga.  No ano de 2008 durante conferência AGILE '08 sobre práticas ágeis, realizada em Toronto, Patrick Debois e Andrew Shafer apresentaram o trabalho construído e também criaram uma lista denominada \textbf{agile-sysadmin}, que inicialmente foi disseminada na Europa para discutir a adoção de metodologias ágeis na infraestrutura e entendermo como equipes de operação poderiam trabalhar no modelo ágil, acompanhando a equipe de desenvolvimento. Isto desencadeou uma série de estudos, confeencias e iniciativas sobre o assunto, que engajou a comunidade à torná-lo popular. 
Um outro marco importante para o DevOps foi a conferência Velocity da O'Really, realizada em 2009, aonde foi apresentado o trabalho \textbf{10+ Deploys per day: Dev and Ops Cooperation at Flickr}, por John Allspaw e Paul Hammond. Este trabalho foi um estudo de caso sobre a capacidade de implantação de mudanças no Flickr após pôr em prática a colaboração entre as equipes, priorizando a entrega de software de qualidade de maneira rápida e eficiente.
Esse estudo foi um marco para alavancar o movimento e nesta mesma conferência, surgiu a ideia de realizar o evento \textbf{DevOpsDays}, que é realizados em diversos países com grupos locais que tem como objetivo disseminar a cultura DevOps.

\subsubsection{O que é DevOps ?}

\textbf{DevOps} é um termo que se tornou uma "buzzword" no mercado e  por não ser prescritivo, muitas empresas tem apresentado a definição que lhes parece correta, bem como alguns profissionais que se intitulam "DevOps", um fenômeno que pode ser descrito conforme a figura 2.

\begin{figure}[h]
    \centering
    \includegraphics[scale=.5]{imagens/devops_elefant.png}
    \caption{O elefante DevOps}
    \label{fig:elefante DevOps}
\end{figure}


Diferente das metodologias ágeis que possuem um manifesto ou ITIL que possui uma descrição sobre seus processos, não existe uma descrição exata sobre como "ser DevOps", pois ela é uma cultura definida por ideias e não por uma definição rigorosa e segundo \citetexto{Davis} está em constante evolução, processos e ideias tem sido discutidos ao longo dos últimos anos e pelo que se observa na comunidade, a tendência é de que permaneça assim ao longo dos próximos anos.  \newline 


A definição proposta por \citetexto{Jabbari2016} é de que: DevOps é uma metodologia de desenvolvimento que busca eliminar os gap's existentes entre Desenvolvimento e Operações. Também pode ser dito como um paradigma, método ou conjunto de princípios e práticas que possibilitam uma melhor comunicação e colaboração, resultando em um trabalho em equipe eficiente entre as equipes de desenvolvedores e operadores.

Um acrônimo que é frequentemente citado nos grupos sobre DevOps é o CALMS, que significa \textbf{C}ulture, \textbf{A}utomation, \textbf{L}ean, \textbf{M}easurement and \textbf{S}haring. Ele é um framework muito útil e é frequentemente mencionado para demonstrar que DevOps não é apenas uma coisa única (como implementar uma ferramenta), mas uma abordagem mais ampla para prestação de serviço, Também trata como mehorar a colaboração, comunicação  e coordenação entre diferentes funções na organização. Segundo \citetexto{Willis}, temos as seguintes ideias sobre cada ponto:
\begin{itemize}
\item \textbf{Culture (Cultura)}: Pessoas e processos primeiro. Se você não tem cultura, todas as tentativas de automação serão inúteis, a mudança no mindset é fundamental para que a cultura prospere. Conforme a experiência de \citetexto{ELBAYADI2014} DevOps não funciona tão bem em estruturas de gerenciamento top-down. Como se trata de uma cultura inclusiva, é fundamental que exista uma aceitação/abertura  para diferentes ideias de diferentes níveis da organização. Uma comunicação abeta é frequentemente discutida como um ponto chave para impulsionar a cultura Devops. Walls (2013) diz: " A cultura DevOps foi criada por muita discussão e debate. Tradicionalmente, equipes técnicas interagem através de sistemas de chamados complexos e com procedimentos que podem ser considerados quase que rituais, o que às vezes requer intervenção da própria diretoria" (p.5);
\item \textbf{Lean (Enxuto)}: No contexto de software, podemos pensar em "Enxuto" o ato de eliminar atividades de baixo valor e avançar de maneira mais ágil, ainda mais relevante no contexto de DevOps, são os conceitos de melhoria contínua e aceitação de falhas. Presente na mentalidade de DevOps, a busca de oportunidades para melhoria contínua está sempre presente, algumas atitudes como retrospectivas de conhecimento regulares, testes A/B de diferentes abordagens de integração para novos usuários do seu produto passam a ser interessantes.  \cite{Atlassian2018}.
\item \textbf{Automation (Automação)}: Este é um dos pontos de partida e mais discutidos para você entender cultura. Neste ponto, as ferramentas começão a formar uma "malha" de automação para DevOps. Ferramentas de gerenciamento de releases, provisionamento, gerenciamento de configuração, integração de sistemas, monitoramento e controle, e ferramentas de orquestração são importantes pedaços desta malha;
\item \textbf{Measureament (Medidas)}: Se você não pode medir, você não pode melhorar. Uma implantação bem sucedida de DevOps mede tudo, o mais frequentemente possível. O core do monitoramento, é ser transparente sobre indicadores de performance que fazem a diferença. 
\item \textbf{Sharing (Compartilhar)}: Compartilhar é o "loopback" do ciclo CALMS. Criar uma cultura onde pessoas compartilham ideias e problema é o ponto crítico. Um outro ponto interessante é a maneira como histórias de sucesso estão sendo compartilhadas para ajudar a comunidade. É um ponto interessante pois além de atrair novos talentos, existe a crença de que, expondo ideias, você pode criar um excelente feedback aberto que, no final, ajuda a melhorar.
\end{itemize}
 
É possível observar que DevOps compartilha diversas características com o movimento ágil, especialmente sobre o foco em pessoas, interações e colaboração. Segundo \citetexto{Davis}, embora DevOps tenha crescido em torno dos princípios do movimento ágil, ele é um movimento cultural separado, mergulhado na história da engenharia de software com um alcance mais amplo, do que a inclusão apenas de desenvolvedores. DevOps estende ideias ágeis e as aplica a uma organização inteira, não só ao processo de desenvolvimento.  

\subsection{Quais são as práticas que DevOps suporta? }
Para que os times de Dev e Ops colaborem de maneira efetiva, existem alguns conjuntos de práticas que automatizam processos de entrega, build e provisionamento e os tornam mais confiáveis e menos suscetíveis à falhas humanas. Algumas das práticas de automação mais difundidas entre DevOps são: \textbf{Integração Contínua, Entrega Contínua, Testes Contínuos, Monitoramento Contínuo, Melhoria Contínua
e Infraestrutura como código}.

\subsubsection{Integração contínua}

Integração contínua é um termo oriundo da metodologia ágil XP (eXtreme Programming) e utilizado em diversas metodologias. Consiste em: Integrar código alterado e/ou desenvolvido ao projeto principal na mesma frequência com que as funcionalidades são desenvolvidas, podendo ser realizado diversas vezes ao dia, ao invés de apenas uma vez. Segundo \citetexto{Fowler2006} a ideia principal por trás deste processo é verificar se as alterações ou novas funcionalidades não criaram novos defeitos no projeto já existente e cada integração é verificada por um build automatizado (incluindo testes) para detectar erros de integração o mais rápido possível, muitos times acreditam que esse tipo de abordagem leva a uma significante redução nos problemas de integração e permite que um time desenvolva software coeso mais rapidamente. As informações sobre builds, testes e implantação são armazenadas em um pipeline de implantação. Neste pipeline de implantação, constam todos os dados referentes às fases do andamento da implantação. Para utilizar integração contínua, é obrigatório adotar algumas práticas, tais como: utilizar controle de versão, usar builds automatizados, usar estes isolados, realizar commits diários no repositório, utilizar um servidor de integração, executar testes automatizados e testes de infraestrutura. É interessante observar que diferente da integração "comum", aonde o software é considerado como não funcional até que a fase de testes valide que o mesmo funcione, já na integração contínua, o software é considerado funcionado, e a cada mudança no software, um conjunto de testes automatizados é executado para garantir seu funcionamento, permitindo assim, em caso de problemas, observar as falhas existentes de maneira mais ágil, diminuindo os prejuízos de correção.  
O objetivo desta prática é resolver problemas usuais de desenvolvimento de software de maneira mais rápida, pois integrando de maneira contínua, o processo se torna mais fluído, o que facilita na rastreabilidade de um problema. Organizações que não possuem este tipo de prática, tendem a realizar integrações após longos períodos de tempo, fazendo com que a rastreabilidade de problemas e o tempo de resolução cresça consideravelmente. Em grande parte dos casos, o servidor de integração contínua também é responsável por aplicar testes automatizados e validar os check-ins entregues por desenvolvedores, também pode realizar testes de integração, desempenho e carga, caso o resultado destes seja positivo, a nova versão do software pode ser disponibilizada para o ambiente de produção.

\subsubsection{Entrega contínua}
O objetivo da prática de \textbf{entrega contínua} é permitir que novas funcionalidades de software sejam entregues para clientes e usuários da maneira mais breve possível, trata-se de uma prática complementar à integração contínua, possibilitando a criação de pipelines de implantação automatizados. \cite{Sharma2014}. \newline
Segundo \citetexto{Humble2012} a repetitividade e confiabilidade são fundamentais para a entrega contínua de software. Eles são obtidos através da automatização de todo processo , desde compilar e utilizar controle de versão para configuração, até a implantação e testes da aplicação, ou seja, a maior parte dos processos e práticas abordados em DevOps são direcionados para viabilizar as práticas de integração contínua e a entrega contínua de software. A automação é um fator crucial para esta prática, pois por meio dela, é que será possível efetuar mudanças entre estágios de criação, implantação, testes e release, realizando a mudanças entre estes como por exemplo, pressionando apenas um botão. 
Para \citetexto{Humble2012}, esta prática possibilita aos times de desenvolvimento a entrega de software em ambientes de produção de maneira previsível, confiável e com poucos riscos ao negócio. Conforme dito esta prática visa encurtar o tempo entre realizar alterações em uma aplicação e conseguir implementá-la, garantindo que possíveis falhas sejam identificadas o mais breve possível, o que torna mais fácil de corrigi-las. \newline
\citetexto{Wotton} destacam que a entrega contínua de software tem como por objetivo:
\begin{itemize}
\item Entregar software de uma maneira rápida e frequente, agregando valor ao negócio e obtendo feedbacks de maneira mais breve possível;
\item Propiciar um aumento na qualidade, estabilidade  e minimizar o downtime de aplicações;
\item Reduzir os riscos do processo de release, testando todos os produtos em ambientes de testes e ambientes semelhantes ao ambiente produtivo;
\item Reduzir o desperdício de trabalho e aumentar a eficiência no processo de desenvolvimento;
\item Entregar e manter o software em um estado de pronto, onde é possível implantar o mesmo de acordo com as necessidades do cliente.
\end{itemize}
Ainda segundo \citetexto{Wotton}, para atingir esses objetivos é  necessário automatizar as seguintes áreas e práticas:
\begin{itemize}
\item  Compilação e empacotamento: O processo de compilação que transforma código fonte em artefatos de implantação;
\item  Builds e Entrega Contínua: A partir da integração contínua são disparados os processos do pipeline de entrega, como testes, implantação e gerenciamento de releases;
\item Automação de testes: Usualmente realizado por um servidor de integração (CI) a cada checkin realizado por desenvolvedores. O conjunto de testes devem cobrir os mais variados níveis de abstração (unidade, integração, aceitação, carga, performance e fumaça). Os testes são distribuídos conforme a necessidade dos estágios do pipeline de implantação, com o objetivo de identificar problemas o mais breve possível;
\item Implantação automática: Implantar software de maneira automatizada, de modo que os times possam realizar esta tarefa de maneira self-service, implantando os sistemas sem dependências da equipe de operações. Este ponto é chave para a entrega contínua;
\item Infraestrutura Como  código: Ferramentas de gerenciamento de configuração para infraestrutura permitem definir e controlar a infraestruturade forma controlada, permitindo a construção de ambientes consistentes, confiáveis e reproduzíveis de maneira automática, gerando assim a base para um pipeline de implantação confiável;
\item Pipeline de implantação: É outro ponto chave para o processo de entrega contínua, ele fornece a visibilidade necessária sobre o andamento de todo o processo de entrega nos estágios e etapas que os releases candidates sofrem, ou seja, o caminho realizado da etapa de código fonte até a produção. Forneem também critérios para um build mover-se entre os estágios do pipeline e conhecimento para a tomada de decisão de acordo com os resultados obtidos.
\end{itemize}


\subsubsection{Testes contínuos e automatizados}
Todos os processos em DevOps produzem código para a produção, assim que a etapa de desenvolvimento é finalizada. Isso requer que o código seja continuamente integrado e em paralelo, é necessário aplicar conjunto de testes que trabalhe com a integração destas novas funcionalidades. Com implantações sendo realizadas de maneira mais rápida e frequente, é impossível executar testes manuais  de todas funcionalidades a cada release. Testar é uma atividade multifuncional que envolve todo o time e deve ser executada continuamente, durante todo o ciclo de vida do desenvolvimento. \cite{Humble2012}.  
Para \citetexto{Duvall2007} não existe integração contínua sem a implementação de testes contínuos e automatizados, pois é por meio deles que os desenvolvedores e demais partes envolvidas no processo de desenvolvimento adquirem confiança sobre as mudanças realizadas realizadas no software.

\subsubsection{Pipeline de implantação}
\citetexto{Braga2015}


\subsubsection{Infraestrutura Como Código}
Infraestrutura como um código é uma abordagem para automação de infraestrutura baseada em práticas de desenvolvimento de software. A ênfase desta prática é em automação de rotinas consistentes e repetitivas, para provisionamento e mudança de configurações em sistemas. As mudanças são feitas em arquivos de definição e após isso, e estas são realizadas de maneira autônoma, incluindo a validação destas alterações.
A premissa das ferramentas modernas é a de que a infraestrutura pode ser tratada exatamente como um software.  Como resultado, todo o trabalho manual que era realizado, como por exemplo, a realização de alterações feitas diretamente em um servidor (ex: Configuração de uma variável de ambiente) deixa de ser realizado. Todas as alterações a serem realizadas devem estar em arquivos de configuração versionados em um repositório como Git ou similar, como qualquer outro código fonte. \citep{Morris2016}. \newline
À partir destes arquivos de definição, um ambiente pode ser criado automaticamente do zero, com as mesmas características de outros ambientes da mesma versão. A qualquer momento, alguém pode olhar o histórico dos arquivos de definição e analisar as configurações do ambiente em uma determinada versão, o que de fato já pode ser considerado uma forma de "documentação viva" que corresponde ao exato estado do sistema naquele ponto.	\newline 
Baseado neste paradigma, criar e destruir ambientes frequentemente é uma boa prática. Um dos motivos é garantir que o ambiente sempre estará consistente com a configuração descrita. Além disso, reforça a cultura de que o ambiente de execução é descartável e não deve ser considerado uma "fonte de verdade" (source of truth).
Um benefícios de transformar a infraestrutura em código é a possibilidade de implementar a integração contínua de maneira adequada, permitindo a execução automática, isolada, e com testes de integração em ambientes com configurações idênticas à de produção. Também torna mais fácil o \textit{rollout} de uma aplicação do ambiente de homologação para o ambiente produtivo e como o processo é bem definido e bem descrito, podemos pensar na escalabilidade de uma aplicação em momentos de pico, de maneira automatizada. Em caso de problemas, o esforço é menor, pois para realizar o \textit{rollback} para a versão anterior da aplicação, seria necessário somente utilizar os arquivos de descrição. \newline
Como principal desvantagem, podemos pensar que para versionar a configuração, será exigido um tempo maior. A instalação manual de um servidor geralmente é rápida, porém, não é escalável. Já a definição da infra em arquivos de definição pode ser mais custosa, principalmente se o operador não for experiente no assunto.

\subsection{Trabalhos relacionados}

%=======================================================================
% Metodologia
%=======================================================================
\section{Metodologia}
A metodologia empregada neste trabalho de pesquisa é um survey do tipo descritivo. \cite{Miguel}. Neste tipo de survey, busca-se o entendimento da relevância de certo fenômeno, bem como descrever a distribuição do fenômeno da população, no caso deste survey, obter "Um panorama sobre experiências e práticas de profissionais em ambientes DevOps". O objetivo primário não é desenvolvimento ou
teste de teorias, mas possibilitar fornecer subsídios para a construção de teorias ou refinamentos. Desta forma, requer a definição de questões a serem endereçadas com argumentação lógica para a escolha da amostra. Tais questões seguem detalhadas na seção Projeto do Survey.
Este survey organiza-se nas seguintes etapas de pesquisa: (i) base teórica, (ii) projeto do survey, (iii) teste piloto e coleta de dados, (iv) análise de dados e síntese.

\subsection{Base teórica}
A etapa de conceituação teórica define os elementos introdutórios além de outros elementos fundamentais deste trabalho de pesquisa. Nesta etapa define-se os conceitos sobre DevOps


\subsection{Trabalhos relacionados}
O trabalho realizado por \citetexto{Kerzazi2016} realiza o mapeamento de atividades e áreas de conhecimento consideradas importantes para profissionais que atuam como "DevOps / Release Engineers", este mapeamento foi realizado em Setembro de 2015, a partir da análise de 211 postagens de emprego realizadas no site monster.com, considerado um dos maiores sites de anúncios de emprego no mundo. Esta análise escolheu, de maneira randômica, 119 postagens de emprego realizadas nos Estados Unidos, 33 no Canadá, 49 no Reino Unido, 6 na Austrália e 4 na índia.  De maneira resumida, os autores realizaram a análise de maneira qualitativa manual de 144 postagens, para entender melhor as responsabilidades e deveres desejados no perfil destes profissionais. Os resultados desta análise foram utilizados para a elaboração de parte do questionário deste presente estudo. As perguntas \textbf{22 e 23} são compostas a partir das habilidades mapeadas por este trabalho.
Para a construção sobre o que é DevOps e quais são suas práticas, 

%=======================================================================
% Resumo em língua estrangeira (sim, é aqui mesmo).
%
% O idioma usado aqui deve necessariamente aparecer nos parâmetros do
% \documentclass, no início do documento.
%=======================================================================


%=======================================================================
% Referências
%=======================================================================


\bibliography{TCC_DevOps}



\end{document}
